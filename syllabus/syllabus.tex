\documentclass[12pt]{article}
\usepackage[margin=1in,letterpaper]{geometry}
\usepackage[utf8]{inputenc}
\usepackage[T1]{fontenc}
\usepackage{graphicx}
\usepackage{amssymb}
\usepackage{amsmath}
\usepackage{palatino}
\usepackage{mathpazo}
\usepackage{color}
\usepackage{hyperref}
\usepackage{multirow}
\usepackage{braket}
\usepackage{relsize}
\usepackage{color, colortbl}
\usepackage{booktabs}
\usepackage[dvipsnames]{xcolor}
\definecolor{darkblue}{RGB}{46,48,147}
\hypersetup{colorlinks=true,
            linkcolor=darkblue,
            urlcolor=darkblue,
            citecolor=darkblue}
\definecolor{Gray}{gray}{0.9}
\definecolor{LightCyan}{rgb}{0.88,1,1}
\definecolor{LightRed}{rgb}{1,0.92,0.92}
\newcommand{\solColor}{blue}
\newcommand{\sol}{\color{\solColor}}
\newcommand*\publistbasestyle{phys}
\usepackage[style=publist,
biblabel=brackets,
sorting=dt,
plauthorhandling=highlight,
nameorder=given-family,
]{biblatex}
\DeclareSourcemap{
 \maps[datatype=bibtex,overwrite=true]{
  \map{
    \step[fieldsource=Collaboration, final=true]
    \step[fieldset=usera, origfieldval, final=true]
  }
 }
}
\renewbibmacro*{author}{
  \iffieldundef{usera}{
    \printnames{author}
  }{
    \printfield{usera} Collaboration
  }
}
\addbibresource{syllabus.bib}
\begin{document}

\begin{center}
  \textbf{
    University of California San Diego\\
    Department of Physics\\
    Physics 141/241, Spring 2023\\
    Computational Physics I: Probabilistic Models and Simulations (4 units)
  }
\end{center}

\noindent\textbf{Instructor}: Javier Duarte, \href{mailto:jduarte@ucsd.edu}{jduarte@ucsd.edu}, OH MWF 3-4pm, MYR-A 5513, \href{https://ucsd.zoom.us/j/93760926244}{Zoom: 93760926244}\\
\noindent \textbf{Teaching assistant}: Yi Guo, \href{mailto:yig053@ucsd.edu}{yig053@ucsd.edu}, OH During Lab, \href{https://ucsd.zoom.us/j/97611399908}{Zoom: 97611399908}\\

\noindent\textbf{Course webpage}:\\
\hspace*{1cm}Login through \href{http://canvas.ucsd.edu}{http://canvas.ucsd.edu}. 
All assignments will be due through Gradescope.\\

\noindent\textbf{Schedule}:
\begin{center}
  \begin{tabular}{|l|c|l|m{90mm}|}
    \hline
    Lecture & MWF  & 2:00--2:50p    & WLH 2208, \href{https://ucsd.zoom.us/j/96620851378}{Zoom: 96620851378}   \\\hline
    Lab     & TuTh & 11:00a--12:20p & MYR-A 4623, \href{https://ucsd.zoom.us/j/97611399908}{Zoom: 97611399908} \\\hline\end{tabular}
\end{center}

\noindent\textbf{First lecture}: Monday, April 3, 2023.\\ 
\textbf{First lab}: Tuesday, April 11, 2023.

\begin{center}
  \rule{\textwidth}{0.5pt}
\end{center}

\noindent\textbf{Textbook}: There is no required textbook for this course.
At the end of the syllabus, we list a bibliography of (mostly free) textbooks and online resources we will draw from.

\begin{center}
  \rule{\textwidth}{0.5pt}
\end{center}

\noindent\textbf{Course information}: This course is an upper-division undergraduate course and introductory graduate course on computational physics, focusing on large-scale deterministic simulations of collective systems.
Basic knowledge of calculus, Newtonian mechanics, Linux, and programming in some language is expected.

The course structure will consist of weekly lectures on conceptual topics, e.g, Newtonian mechanics, and lab sections on computational tools, e.g., programming in Python and C/C++.
Students will learn how to apply physical reasoning to programming, optimize and debug code, create simulations of physical systems.
We will focus primarily on solving the $N$-body problem: predicting the individual motions of a group of celestial objects interacting with each other gravitationally.
Through this problem, we will study increasingly accurate methods, including Euler, Runge-Kutta, leap-frog integration, tree-code methods, and Dehnen's algorithm (i.e., gyrfalcON).
Students will also learn how to use modern tools to efficiently solve scientific computing problems interpreted (Python) vs. compiled (C/C++) languages and how to link the two.
There will be 3 homework assignments.
There will also be a midterm and final project in which students will work in groups to reproduce the motion of one of 3 observed systems.


\begin{center}
  \rule{\textwidth}{0.5pt}
\end{center}

\noindent\textbf{Student learning outcomes}: Upon successful completion of Physics 141/241, students will be able to:
\begin{itemize}
  \item Design computer programs to numerically solve physics problems, like the $N$-body problem.
  \item Consider multiple approaches and compare their computational performance, accuracy, and fidelity to physical laws.
  \item Find and choose the best tool or programming language for the task.
  \item Visualize the solutions.
  \item Collaborate with peers to tackle complex, realistic problems.
  \item Present findings.
\end{itemize}

\begin{center}
  \rule{\textwidth}{0.5pt}
\end{center}

\noindent\textbf{Grading policy}: Your final course grade will be determined according to the following:
\begin{itemize}
  \item 40\% Homework.
  \item 10\% Quizzes.
  \item 5\% Participation in class, via Slack, and completion of exit tickets.
  \item 20\% Midterm project.
  \item 25\% Final project.
\end{itemize}

\begin{center}
  \rule{\textwidth}{0.5pt}
\end{center}

\noindent\textbf{Drop policy}: The lowest homework score is dropped automatically.
This drop policy is designed to account for any and all illnesses, family, medical, mental, or other emergencies.

If you have an extended emergency (e.g., a long hospital stay) that hinders your ability to turn complete assignments beyond the emergency policy allowance, contact the professor directly as soon as the situation arises.

\begin{center}
  \rule{\textwidth}{0.5pt}
\end{center}

\noindent\textbf{Discussion board}: We will use Slack: \href{https://join.slack.com/t/ucsdphys139/shared\_invite/zt-110gwd4lx-pZBsItfcxhbOD5BV6afVDA}{ucsdphys139.slack.com}

\begin{center}
  \rule{\textwidth}{0.5pt}
\end{center}


\noindent\textbf{Homework}: Homework assignments will be submitted as code and a report on Gradescope.

There will be a first deadline to submit a ``draft'' version of the homework, which will be graded based on effort and completeness.

There will be a second deadline to submit a ``corrected'' version of the homework, which will be graded based on effort and correctness.

\begin{center}
  \rule{\textwidth}{0.5pt}
\end{center}

\noindent\textbf{Midterm \& final projects}:
For the midterm and final projects, students will work in groups of $\sim$4 to reproduce the motion of an observed system.
The project deliverables are: (1) code provided as a public GitHub repository, (2) a 20-minute presentation by all members of the group, and (3) self and peer evaluations for group contributions.

\begin{center}
  \rule{\textwidth}{0.5pt}
\end{center}

\noindent\textbf{Attendance (lectures and labs)}: In-person lecture attendance is not required, but strongly recommended.
Lecture will be mostly conceptual while labs will include hands-on portions, with interactive problem-solving and pair programming throughout.
These sessions will be recorded.

\emph{Exit tickets}: At the end of each class, you will be invited to fill out an \href{https://forms.gle/b7ZDZRm1czrnHaGBA}{exit ticket}.

\begin{center}
  \rule{\textwidth}{0.5pt}
\end{center}

\noindent\textbf{Academic integrity}: Please read the College Policies section of the \href{http://senate.ucsd.edu/Operating-Procedures/Senate-Manual/Appendices/2}{UCSD’s Policy on Integrity of Scholarship}.
These rules will be enforced.
Cheating includes, but is not limited to: submitting another person's work as your own, copying from any person/source, and using any unauthorized materials or aids during exams.

For homework assignments, copying from an online solution, a peer's solution, a Chegg solution, or shared work (on Discord, for example) is considered cheating.
Collaboration is encouraged, but by the time you start writing your own solution to turn in, you should not be looking at any other source.
You should know the rough outline of the solution well enough that you do not need to reference something line-by-line.
Plagiarizing a solution but changing variable names is considered cheating.
Soliciting help online via Chegg, Quora, etc. is considered cheating.
If suspected, you might be asked to rework similar problems in a Zoom one-on-one meeting with the instructor and/or TA.

Any questions on what constitutes an academic integrity violation should be addressed to the instructor; any violation of academic integrity will result in immediate reporting to the UCSD Office of Academic Integrity, and can result in an automatic ``F'' for the course at the discretion of the instructor.

\begin{center}
  \rule{\textwidth}{0.5pt}
\end{center}

\noindent\textbf{Counseling and Psychological Services (CAPS):} The mission of CAPS is to promote the personal, social, and emotional growth of students.
Many services are available to UCSD students including individual, couples, and family counseling, groups, workshops, and forums, consultations and outreach, psychiatry, and peer education.
To make an appointment, call (858) 534-755.
For more information, visit \href{https://wellness.ucsd.edu/caps/}{https://wellness.ucsd.edu/caps/}.

\begin{center}
  \rule{\textwidth}{0.5pt}
\end{center}

\noindent\textbf{\emph{Schedule}} (Subject to change):\\

\noindent\textbf{Week 1}

\emph{Monday 4/3}: \underline{Lecture}: Course overview, galaxy collisions, preview of the final projects

\emph{Tuesday 4/4}: \textbf{No class}

\emph{Wednesday 4/5}: \underline{Lecture}: Newtonian, Hamiltonian, Lagrangian mechanics

\emph{Thursday 4/6}: \textbf{No class}

\emph{Friday 4/7}: \underline{Lecture}: Numerical methods: Euler, Runge-Kutta, Verlet, leapfrog integration

\noindent\textbf{Week 2}

\emph{Monday 4/10}: \underline{Lecture}: Numerical methods (continued): 

\emph{Tuesday 4/11}: \underline{Lab}: Python, Jupyter, and DataHub

\emph{Wednesday 4/12}: \underline{Lecture}: Phase space, Hamiltonian mechanics, symplectic integrators

\emph{Thursday 4/13}: \underline{Lab}: NumPy, SciPy, Numba, and Matplotlib

\emph{Friday 4/14}: \underline{Lecture}: Gravitational potential and $N$-body equations

\noindent\textbf{Week 3}

\emph{Monday 4/17}: \underline{Lecture}: Orbits in spherical potentials, virial parameters

\emph{Tuesday 4/18}: \underline{Lab}: Introduction to C/C++

\emph{Wednesday 4/19}: \underline{Lecture}: Collisionless Boltzmann equation (CBE); \textbf{Homework 1 due}

\emph{Thursday 4/20}: \underline{Lab}: Introduction to C/C++ (continued)

\emph{Friday 4/21}: \underline{Lecture}: CBE (continued)

\noindent\textbf{Week 4}

\emph{Monday 4/24}: \underline{Lecture}: Plummer's sphere

\emph{Tuesday 4/25}: \underline{Lab}: Nemo + GLNemo2 tutorial

\emph{Wednesday 4/26}: \underline{Lecture}: Plummer's sphere (continued)

\emph{Thursday 4/27}: \underline{Lab}: Nemo tools (concatenating, splitting, rotating, displacing snapshots); How to read documentation

\emph{Friday 4/28}: \textbf{No class}

\noindent\textbf{Week 5}

\emph{Monday 5/1}: \textbf{No class}

\emph{Tuesday 5/2}: \underline{Lab}: Midterm project assignment and overview

\emph{Wednesday 5/3}: \underline{Lecture}: Galaxy modeling: isothermal sphere and king model

\emph{Thursday 5/4}: \underline{Lab}: Open session

\emph{Friday 5/5}: \underline{Lecture}: Galaxy modeling: isothermal sphere and king model (continued)

\noindent\textbf{Week 6}

\emph{Monday 5/8}: \underline{Lecture}: 

\emph{Tuesday 5/9}: \underline{Lab}:

\emph{Wednesday 5/10}: \underline{Lecture}: Galaxy modeling: isothermal sphere and king model

\emph{Thursday 5/11}: \underline{Lab}:

\emph{Friday 5/12}: \underline{Lecture}: Galaxy modeling: isothermal sphere and king model (continued)

\noindent\textbf{Week 7}

\emph{Monday 5/15}: \underline{Lecture}: Bulge and halo distribution functions

\emph{Tuesday 5/16}: \underline{Lab}:

\emph{Wednesday 5/17}: \underline{Lecture}: Bulge and halo distribution functions (continued); \textbf{Midterm due}

\emph{Thursday 5/18}: \underline{Lab}:

\emph{Friday 5/19}: \underline{Lecture}: Bulge and halo distribution functions (continued)

\noindent\textbf{Week 8}

\emph{Monday 5/22}: \underline{Lecture}: Final project overview

\emph{Tuesday 5/23}: \underline{Lab}:

\emph{Wednesday 5/24}: \underline{Lecture}: Midterm/final discussion (continued);  \textbf{Homework 3 due}

\emph{Thursday 5/25}: \underline{Lab}:

\emph{Friday 5/26}: \underline{Lecture}: Midterm/final discussion (continued)


\noindent\textbf{Week 9}


\emph{Monday 5/29}: \textbf{No class}

\emph{Tuesday 5/30}: \underline{Lab}:

\emph{Wednesday 5/31}: \underline{Lecture}: Hierarchical tree method

\emph{Thursday 6/1}: \underline{Lab}:

\emph{Friday 6/2}: \underline{Lecture}: GyrfalcON: force algorithm with complexity $\mathcal{O}(N)$


\noindent\textbf{Week 10}

\emph{Monday 6/5}: \underline{Lecture}: 

\emph{Tuesday 5/30}: \underline{Lab}: Team presentations

\emph{Wednesday 5/31}: \underline{Lecture}: 

\emph{Thursday 6/1}: \underline{Lab}: Team presentations

\emph{Friday 6/2}: \underline{Lecture}: 



\begin{center}
  \rule{\textwidth}{0.5pt}
\end{center}
\nocite{*}

\noindent\textbf{\emph{Bibliography}}:\\
\end{document}
